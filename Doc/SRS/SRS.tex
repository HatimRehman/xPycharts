\documentclass[12pt, titlepage]{article}

\usepackage{booktabs}
\usepackage{tabularx}
\usepackage{hyperref}
\usepackage{mdframed}
\hypersetup{
    colorlinks,
    citecolor=black,
    filecolor=black,
    linkcolor=red,
    urlcolor=blue
}
\usepackage[round]{natbib}

\newmdenv[linecolor=black]{reqbox}

\title{SE 3XA3: Development Plan\\XPyCharts}

\author{Team 4, xPy
		\\ Hatim Rehman (rehmah3)
		\\ Louis Bursey (burseylj)
		\\ Sarthak Desai (desaisa3)
}

\date{\today}

\input{../Comments}

\begin{document}

\maketitle

\pagenumbering{roman}
\tableofcontents
\listoftables
\listoffigures

\begin{table}[bp]
\caption{\bf Revision History}
\begin{tabularx}{\textwidth}{p{3cm}p{2cm}X}
\toprule {\bf Date} & {\bf Version} & {\bf Notes}\\
\midrule
Date 1 & 1.0 & Notes\\
Date 2 & 1.1 & Notes\\
\bottomrule
\end{tabularx}
\end{table}

\newpage

\pagenumbering{arabic}

This document describes the requirements for ....  The template for the Software
Requirements Specification (SRS) is a subset of the Volere
template~\citep{RobertsonAndRobertson2012}.  If you make further modifications
to the template, you should explicity state what modifications were made.

\section{Project Drivers}

\subsection{The Purpose of the Project}

\subsection{The Stakeholders}

\subsubsection{The Client}

\subsubsection{The Customers}

\subsubsection{Other Stakeholders}

\subsection{Mandated Constraints}

\subsection{Naming Conventions and Terminology}

\subsection{Relevant Facts and Assumptions}

User characteristics should go under assumptions.

\section{Functional Requirements}

\subsection{The Scope of the Work and the Product}

\subsubsection{The Context of the Work}

\subsubsection{Work Partitioning}

\subsubsection{Individual Product Use Cases}

\subsection{Functional Requirements}

\section{Non-functional Requirements} %lLouis

\subsection{Look and Feel Requirements}

\subsection{Usability and Humanity Requirements}

\begin{reqbox}
%---
\begin{tabular}{ccc}
Requirement \#: & Requirement Type: 10 & Event/Use case \#: \\
\end{tabular} \\
%---
\textbf{Description:} The product should be easy to implement for novice Python programmers \\
\textbf{Rationale:} The programmer using this library is more productive if they don't have to learn a convoluted library
\textbf{Originator:} Louis Bursey\\
\textbf{Fit Criterion:} 80\% of programmers familiar with Python successfully use the product  \\
%---
\begin{tabular}{ll}
\textbf{Customer Satisfaction:} 5 & \textbf{Customer Dissatisfaction:} 5 \\
\textbf{Priority:} High & \textbf{Conflicts:} None\\
\end{tabular} \\
%---  
\textbf{Supporting Materials:} None \\
\textbf{History:} Created October 5, 2016
%--
\end{reqbox}

\subsection{Performance Requirements}

\subsection{Operational and Environmental Requirements}

\subsection{Maintainability and Support Requirements}

\subsection{Security Requirements}

\subsection{Cultural Requirements}

\subsection{Legal Requirements}

\subsection{Health and Safety Requirements} %end Louis

This section is not in the original Volere template, but health and safety are
issues that should be considered for every engineering project.

\section{Project Issues}

\subsection{Open Issues}

\subsection{Off-the-Shelf Solutions}

\subsection{New Problems}

\subsection{Tasks} 

\subsection{Migration to the New Product}

\subsection{Risks}

\subsection{Costs}

\subsection{User Documentation and Training}

\subsection{Waiting Room}

\subsection{Ideas for Solutions}

\bibliographystyle{plainnat}

\bibliography{SRS}

\newpage

\section{Appendix}

This section has been added to the Volere template.  This is where you can place
additional information.

\subsection{Symbolic Parameters}

The definition of the requirements will likely call for SYMBOLIC\_CONSTANTS.
Their values are defined in this section for easy maintenance.


\end{document}